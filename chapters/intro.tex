
%! TEX root = ../main.tex
\documentclass[main]{subfiles}

\begin{document}

\chapter{はじめに}
\label{cha:intro}

プログラミング学習に置いて、誰かが教示してくれるのを受動的に待つのではなく、自らインプットとアウトプットを能動的に繰り返していくことが大事である。能動的に学習を行う上で、学習中に出力されたエラーメッセージから必要な情報を読み取り、問題解決につなげる「エラー読解力」が重要となってくる。また、現状の課題を理解し的確なワードで情報収集をする「検索力」が重要となってくる。

本研究では、先に述べた「エラー読解力」「検索力」が乏しいプログラミング初学者をサポートするためのWebアプリケーションを開発する。初学者をサポートする機能として、アプリケーションには、解決に必要なエラーメッセージの構成と役割を提示する機能、そして、検索ワードを提案する機能がある。

まず、エラーメッセージの構成と役割を提示する機能では、与えられたエラーメッセージの中から、エラーの概要やエラーの発生したファイル、エラーの詳細文などを抽出する。それらのワードの役割や読み方を順に示すことによって、初学者がエラーメッセージから原因や意味を読み取り、根拠を持ってエラーを解決する力を育む。
次に、検索ワードを提案する機能では、エラーメッセージ、取り組んでいる内容の入力を使用者に求める。入力した内容から検索ワードを提案することで、使用者が求めている情報を得られるようにサポートする。


本論文の構成は以下の通りである。まず \ref{cha:related} 章では、関連研究について述べる。最後に、\ref{cha:conclusion} 章では、本論文のまとめと今後の課題について述べる。

\end{document}
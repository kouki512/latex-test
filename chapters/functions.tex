
\documentclass[main]{subfiles}

\begin{document}
\chapter{アプリケーションの各機能}
\section{エラーメッセージから必要な情報を抽出する機能}
エラー解決をするにあたって理解すべき情報を、どのようにエラーメッセージから抽出しているかについて述べる。具体的な、入力から単語を抽出する実装方法や、Railsのエラー解決を行う上で、どんな情報を理解すべきかについてもこのセクションで述べる。

\section{エラーの構成と意味を提示する機能}
エラーメッセージの見るべき場所や、そこからどんな情報が得られるかを説明している箇所について述べる。具体的には、各項目の説明を追加した意図や、各項目の説明により使用者が理解できるようになることについて述べていく。

\section{検索ワードの提案する機能}
効率的に情報を得られる検索ワードや、それを採用した理由について述べる。

\end{document}
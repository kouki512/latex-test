%第三章 <= ハードコーディングしない....
\documentclass[main]{subfiles}

\begin{document}
\chapter{エラー解決にあたって理解すべき情報}
エラーメッセージの中から、エラー解決をするにあたって理解すべき情報について述べる。なお、ここではRuby on Railsのエラーを一例に挙げる。具体的には、SyntaxError等エラーの大枠、エラーの詳細文、エラーが発生しているパスなどを列挙し、それぞれの項目を理解すべき理由について述べる。

% ここに書いてある内容は、一般論的な内容を、Rails を例にしてやろうとしている。
% しかし、ここで話されるのは、Rails のエラーの枠組みをもろにつかっている。
% これをそのままやりたかったら、他の言語もこうなってますという例示が必要。
% Rails のエラーの枠組みが、広く他の言語でも通用するならいいけど、そうじゃなければ成立しない。
% おそらく、フレームワークかそうじゃないか?、さらに、他のフレームワークでも成立しない話のような。。。

\end{document}


%! TEX root = ../main.tex
\documentclass[main]{subfiles}

\begin{document}

\chapter{todo}
\label{cha:intro}
    卒研の進捗を確認する章
\begin{itemize}
    \item 章立てを考える (12/2 17時締め切り)\\
        章タイトルを決めて、何を書く章なのかを数行で書く
        一つの章につき1ファイルで構成しているので、新しい章をごとにファイルを作成する。(chaptersフォルダ)
        
    \item 第一章:はじめに
    \mbox{}\\ 既に記載しているイントロダクション
    \item 第二章:エラー解決に必要な情報
    \mbox{}\\ エラー解決をするにあたって理解すべき情報について述べる。 具体的には、SyntaxError等エラーの大枠、エラーの詳細文、エラーが発生しているパスなどを列挙し、それぞれの項目を理解すべき理由について述べる。
    \item 第三章:エラーメッセージから必要な情報を抽出する機能
    \mbox{}\\ 第二章で述べた、エラー解決に必要な情報をどのようにエラーメッセージから抽出しているかを述べる。具体的には、エラーメッセージの入力から、正規表現を用いた単語の抽出方法の一連の流れについて述べていく。
    \item 第四章:エラーの構成と意味を提示する機能
    \mbox{}\\ エラーメッセージの見るべき場所や、そこからどんな情報が得られるかを説明している箇所について述べる。具体的には、各項目の説明を追加した意図や、各項目の説明により使用者が理解できるようになることについて述べていく。
    \item 第五章:検索ワードの提案する機能
    \mbox{}\\ 効率的に情報を得られる検索ワードや、それを採用した理由について述べる。
    
    \item 第六章:有効性の検証\mbox{}\\ 作成したアプリケーションにより、使用者がどのくらいエラーの解決ができるようになったの検証結果について述べる。
    \item 第七章:まとめと今後の課題
    \mbox{}\\ 使用者が更にエラーについての理解を深めるために、どのような機能追加が今後必要かについて述べる。

    
\end{itemize}
開発の進捗を確認する章
    \begin{itemize}
        \item 残りの作業をリストアップする 12/2 17:00締め切り \\
            開発の残りの作業を書く
        \item 文章全体から、説明していく文章を強調して表示する機能(作成中)
        \item アプリケーションの有効性の検証(12/2にプレゼミ生に依頼を出し来週あたりに実施)
        \item エラーの詳細に対する解説を追加(出来るところまで。代表的なエラーは追加したので時間が余ったら実施)
    \end{itemize}
\end{document}
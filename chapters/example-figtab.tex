%! TEX root = ../main.tex
\documentclass[main]{subfiles}

\begin{document}

\chapter{図と表}
\label{cha:exfig}

\section{図の貼り方}

図を一枚だけ貼る場合は、図 \ref{fig:one-fig} の貼り方を参照し、
二枚貼る場合は、図 \ref{fig:two-fig} の貼り方を参照せよ。
基本的にまず、それぞれの該当箇所をコピー&ペーストして、図の実態がある場所 (例\verb|../figures/dummy1.png|)
を修正し、さらに、キャプショ \verb|\caption{...}|、\verb|\label{}|を修正したらよい
論文に貼る図として、pdf 、png、または、jpg 形式のファイルを準備せよ。これらが綺麗な図となる。


\begin{figure}[tb]
    \begin{center}
        \fbox{
            \includegraphics[width=.8\linewidth]%
            {../figures/dummy1.pdf}
        }
        \caption{一枚だけ表示する場合}
        \label{fig:one-fig}
    \end{center}
\end{figure}

\begin{figure}[tb]
    \begin{minipage}[b]{\linewidth}
        \centering
        \fbox{
            \includegraphics[width=.8\linewidth]{../figures/dummy2.png}
        }
        \subcaption{dummy2.png}
    \end{minipage}\\
    \begin{minipage}[b]{\linewidth}
        \centering
        \fbox{
            \includegraphics[width=.8\linewidth]{../figures/dummy3.jpg}
        }
        \subcaption{dummy3.jpg}
    \end{minipage}
    \caption{二枚を表示する場合}
    \label{fig:two-fig}
\end{figure}


\section{表の書き方}

表 \ref{tab:example} をベースに作成したらよい。修正すべき点は、図の場合とほぼ同じ。

\begin{table}[tb]
    \caption{センサノードのパラメータ}
    \label{tab:example}
    \centering
    \begin{tabular}{lc}
        \hline
        Frequency                     & 2.44 [GHz]   \\
        Transmission rate             & 250 [Kbit/s] \\
        Antenna gain                  & 2 [dBi]      \\
        Minimum receiving sensitivity & -96 [dBm]    \\
        Voltage                       & 3.3 [V]      \\
        Size per packet               & 127[Byte]    \\
        \hline
    \end{tabular}
\end{table}

\end{document}
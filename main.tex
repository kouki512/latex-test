\documentclass[a4paper, 11pt, uplatex]{oecu-thesis}

\usepackage{subfiles}
\usepackage{listings}
\usepackage{plistings}
\usepackage{color}
\usepackage[dvipdfmx]{graphicx}
\usepackage{url}
\usepackage{siunitx}
\usepackage{enumerate}
\usepackage{paralist}
\usepackage{amsmath,amssymb}
\usepackage{mathtools}
\usepackage{times}
\usepackage[newfloat]{minted}
\usepackage[hang,small,bf,labelsep=space]{caption}
\usepackage[subrefformat=parens]{subcaption}
\usepackage[dvipdfmx]{hyperref}
\usepackage{pxjahyper}
\usepackage{framed}
\usepackage[dvipdfmx]{pdfpages}
\usepackage[backend=biber,style=ieee]{biblatex}
\addbibresource{references.bib}
\captionsetup{compatibility=false}

\title{プログラミング初学者の検索・エラー解決を補助するwebアプリケーションの開発}
\author{石田 幸暉}
\date{{令和}\rensuji{4}年\rensuji{12}月}
\学生番号{HT20A008}
\指導教員{久松 潤之 准教授}

% 特別研究の場合はコメントをはずす。卒業研究の場合はコメントアウトする。
%\論文種別{特別研究論文}
\年度{令和4}

\所属{総合情報学部 情報学科}


\begin{document}
\maketitle

\subfile{chapters/abstract} % 一番最後に書く
\subfile{chapters/todo} 
\subfile{chapters/intro} 
\subfile{chapters/related} 
\subfile{chapters/information_needed.tex} 
\subfile{chapters/application_overview.tex} 
\subfile{chapters/functions.tex} 
\subfile{chapters/validation.tex} 
\subfile{chapters/conclusion} 
% \subfile{chapters/example-figtab.tex} 画像の貼り方のものなのでいらない
\subfile{chapters/acknowledgements} 
\printbibliography[title=参考文献]

\end{document}